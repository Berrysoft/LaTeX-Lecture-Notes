\documentclass[12pt]{article}
\usepackage{amsmath,amsfonts,amssymb,bm,hyperref}
\author{Qi-an Fu}
\title{Math Formula}
\begin{document}
\maketitle
\tableofcontents
\listoftables

\section{Basic}
This is an inline equation:
$ (x_1 + x_2)^2 = (x_1 - x_2)^2 + 4 x_1 x_2 $ \\
Greek letter:
\[ \alpha, \beta, \gamma \]
\[ \delta, \Delta, \Psi, \Omega \]
Equal sign:
\[ =, \neq, \leq, \geq, \equiv \]
\[ \approx, \ll, \gg \]
Fraction:
\[ \frac{a}{b}, \frac ab \]
Calculus:
\[ \int^a_b, \lim_{n \rightarrow \infty} \]
\[ \sum_{n=0}^{\infty}, \prod_\epsilon \]
Other:
\[ \bar{a}, \overline{a+b}, \underline{a+b} \]
\[ \vec{a}, \overrightarrow{AB} \]
\[ \underbrace{a_1+a_2+\ldots+a_n}_n \]
\[ \overbrace{a_1+a_2+\ldots+a_n}^n \]
\[ \binom{n}{k}, \mathrm{C}_n^k \]
\[ \stackrel{?}{=} \]
\[ \cdot, \cdots, \dots, \ldots \]
\[ \circ, \times \]
$ {\displaystyle \frac ab} $
\section{Font Face}
\[ \mathbb{R}, \mathbf{B}, \boldmath{B} \]
\[ \mathrm{Hi},\ I\ have\ some\ \text{Text}. \]
\section{Equation Environment}
\begin{equation}
  \label{eq1}
  \left\{ \left[ \left( \frac{1}{1+x^2} \middle/ (1 + y) \right) \right] \right\}
  \quad \left. \frac{\mathrm{d}f}{\mathrm{d}x} \right|_{x = 0}
\end{equation}
No auto numbering:
\begin{equation*}
  \int\!\!\!\int f(x, y) \; \mathrm{d} x \mathrm{d} y
  \quad \text{or} \quad \iint
\end{equation*}
\section{Table}
\begin{table}[htbp]
  \begin{center}
    \begin{tabular}{|l|c|r|}
      \hline
      \multicolumn{2}{|c|}{Value} & third \\ \hline
      1           & 2             & 3     \\ \cline{1-1}
    \end{tabular}
    \caption{This is a table}
    \label{tab1}
  \end{center}
\end{table}
\section{Matrix}
\begin{displaymath}
  \mathbf{x} =
  \left(\begin{array}{ccc}
    x_{11} & x_{12} & \ldots \\
    x_{21} & x_{22} & \ldots \\
    \vdots & \vdots & \ddots
  \end{array}\right)
\end{displaymath}
\section{Multiline Equation}
\begin{align}
  a     &= b + c \\
  c + d &= e
\end{align}
\begin{equation}
  \left\{
  \begin{aligned}
    a     &= b + c \\
    c + d &= e
  \end{aligned}
  \right.
\end{equation}
\section{Define myself}
\newcommand{\ud}{\mathrm{d}}
\newcommand{\dif}[2]{\frac{\ud {#1}}{\ud {#2}}}
\[ \dif fx \]
\end{document}
